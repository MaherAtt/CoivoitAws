\documentclass{article}
\usepackage[utf8]{inputenc}
\usepackage{graphicx}
\usepackage[margin=0.5in]{geometry}
\title{Rapport Projet Applications Web Sécurisées}
\author{Maher Attouche - Benammar Ismail - Shade Alao - Keskes Yassmine - Carre Clement}
\date{April 2020}

\begin{document}

\maketitle
\section{Cahier des charges}
\subsection{Introduction}
Tous les jours des milliers d'étudiants utilisent leurs voitures pour aller à l'université. Dans une grande majorité des cas, ces trajets domicile-université se font seuls et conduisent à une congestion des centres urbains dû au dépassement des capacités des collectivités. Cela  entraine donc une répercussion sur l’environnement.  Les études montrent que les embouteillages automobiles ou les transports en commun bondés sont la 1ère cause de stress du déplacement domicile-université pour plus de 72\% des personnes interrogées.

Selon l’ADEME, il suffirait pourtant d’une réduction de 4\% du nombre de voitures en circulation afin de passer de cette situation de congestion et d’embouteillage à celle d’un trafic fluide. Afin de répondre à cette véritable problématique, le covoiturage s’est totalement démocratisé. Partager un véhicule entre plusieurs passagers afin d’effectuer un trajet identique permet effectivement de résoudre ces différents enjeux.


\subsection{Fondement du projet}
\subsubsection{Contexte du projet}
Ce projet est réalisé dans le cadre du module Applications Web et Securité (AWS) du semestre 2 du master 1 informatique de l'UVSQ. Le sujet Réalisation d'une application web de covoiturage entre les étudiants a été choisi par les membres du groupe, notre travail consistera donc a réaliser cette application web.

\subsubsection{But du projet}
Le but du projet est d'apprendre a travailler en groupe et de s'initier au domaine des applications Web et de la sécurité informatique.
L'objectif du projet est donc de réaliser une application web qui facilite le covoiturage entre étudiants. L'application web devra permettre de s’inscrire, se connecter, rechercher ou proposer un trajet, candidater pour un trajet,  accepter ou refuser une demande, accéder aux profils des autres usagers, communiquer avec eux via un système de messagerie, ainsi que donner son avis sur les divers utilisateurs de la plateforme.

\subsection{Contraintes sur le projet}
\subsubsection*{Contraintes temporelles}
Respecter le calendrier du projet (05/03/20  au 30/04/20).

\subsubsection*{Contraintes techniques}
Le BackEnd de l'application ne doit pas être réalisé avec du Php.

\subsubsection*{Environement de fonctionnement}
L'application doit être accessible sur smartphone, tablettes et ordinateurs et optimisé pour ceux-ci.

\subsection{Exigences fonctionnelles}
\begin{itemize}
    \item Créer un compte.
    \item Se connecter.
    \item Modifier les informations de l'utilisateur.
    \item Proposer un nouveau trajet.
    \item Rechercher un trajet.
    \item Postuler et reserver sur un trajet.
    \item Accepter/Refuser les demandes de participation aux trajets.
    \item Accéder aux profils des autres utilisateurs.
    \item Poster des avis et noter les utilisateurs.
    \item Rappel des trajets prévu pour chaque utilisateur.
    \item Suivre l'état des demandes de réservation.
    \item Communiquer avec les autres utilisateurs de l'application à travers une messagerie.
\end{itemize}

\subsection{Exigences non fonctionneles}
\begin{itemize}
    \item L'application doit être sécurisée afin d’empêcher l’accès aux comptes d’autres utilisateurs dans le but d’usurper leur identité ou de voler des données. 
    \item La prise en main du site doit être facile.
    \item L’apparence du site doit être agréable.
    \item Toute action doit être rapide à effectuer.
\end{itemize}

\subsection{Conclusion}
La rédaction du cahier des charges nous a permis de bien définir les caractéristiques de notre application. Nous avons donc présenté dans ce cahier des charges tous les éléments nécessaires à la réalisation du projet. Le produit qui sera livré le 30/04/2020, devra donc être fonctionnel et devra au moins présenter toutes les fonctionnalités citées précédemment.

\section{Conception}
\subsection{Organisation du groupe}
Malgré un échange de rôle en début de projet, les nouvelles contraintes de travail à distance nous ont plus ou moins forcés à garder des rôles stables tout au long du développement de notre site de covoiturage. 
En majeur partie Maher s’occupait du Back-End, en étroite collaboration avec Ismail. 
Shadé ainsi qu’Ismail se sont chargés du Front-End. \\
Yasmine s’est focalisée sur la partie recherche concernant l’aspect sécurité du site, et au moment de son application, elle a également contribué à sa mise en place. \\
Quant à Clément, il a été force de proposition concernant les différentes fonctionnalités à développer. Il a également contribué avec Maher à la vérification du code et aux commentaires.

\subsection{Moyens de communication}
Nous savons tous que la communication dans un groupe est l’un des piliers essentiels pour mener à bien les différents projets. Communiquer régulièrement sur l’avancée de tout un chacun, s’entraider pour débloquer un coéquipier permettent en autre de maintenir une bonne cohésion de groupe et une motivation constante. Ce travail d’équipe a d’autant plus été important durant cette période de confinement puisqu’il n’y avait plus de période de TD  obligatoire hebdomadaire.\\
Pour ceci, nous avons en grande partie utilisé Discord. Cette application nous a permis de rester en contact, de discuter sur les tâches que chacun doit réaliser, de réorganiser des réunions grâce aux canaux vocaux de ce dernier et de partager des fichiers, de la documentation et certaines recherches réalisées. \\
Dans un premier temps glitch (https://glitch.com/edit/#!/bolder-spotless-steed?path=index.html\%3A1\%3A0) nous a permis de partager le projet. Cet outil a été choisi car il est  gratuit, en ligne, simple d’utilisation ou l’on peut voir en direct sur internet le résultat du produit et il permet également la modification du code instantanément par plusieurs personnes, ce qui évite d’avoir des conflits. Mais l’une des faiblesses que nous avons pût rencontrer au fils du temps est sa vitesse d’exécution. En effet les pages mettaient un peu trop de temps à charger. C’est principalement pour cette raison, que nous avons décidés unanimement da basculer sur  GitHub (https://github.com/MaherAtt/CoivoitAws). \\  Nous avons choisi cet outil pour collaborer car il était connu de tous, il est simple d’utilisation mais nous devions constamment veiller à régler correctement tous les conflits lors d’une fusions afin de ne pas effacer le contenu que chacun a pu ajouter au fils du temps.  

\subsection{Les technologies utilisées}
La réalisation de notre application s'appuie sur un ensemble d'outils et de technologies de développement Web. On distingue :
Les éléments de base servant à la création d’un site vitrine : 
\subsubsection{FrontEnd:}
\begin{itemize}
    \item HTML
    \item CSS
    \item Javascript
\end{itemize}
Concernant le développement du front-end, certains membres du groupe avaient déjà utilisés Bootsrap qui est un outil open source servant entre autre à faciliter la réalisation du CSS et à rendre le site responsive.

\subsubsection{BackEnd}
\textbf{NodeJs} : Grace à nos différentes recherches, il nous est paru évident d’utiliser Nodejs pour le développement du site puisqu’il a été en 2019 le framework javascript le plus utilisé (https://www.konstantinfo.com/blog/angularjs-vs-nodejs-vs-reactjs/). \\ Etant étudiants et en formation pour le monde du travail, se former à un outil fréquemment utilisé est donc primordial pour nous. \\
NodeJs nous permet d’utiliser JavaScript côté client et serveur aussi. Comparé à PHP, Nodejs est plus performant et plus rapide côté serveur grâce au moteur d’exécution V8 de Google et à son fonctionnement non bloquant. D’ailleurs c’est ce qui a poussé cette technologie à être employé et éprouvé par les géants du web. \\
\textbf{ExpressJs} :
Nous avons utilisé le framework Express.js pour le routage des différentes pages. \\ 
\textbf{Embedded JS} :
EJS nous sert à appeler du code HTML depuis NodeJS dans le but de faciliter la dynamisation des pages car EJS Permet de créer un template qui sera rempli lors de la reponse du serveur par les informations adequattes. \\
\textbf{MySql} :
Concernant la base de données (BDD), nous avons utilisé le système de gestion de bases de données relationnelles MySQL. \\
\textbf{Heroku} :
L’application a été déployée grâce à Heroku qui est une plateforme permettant le déploiement d’applications en s'appuyant sur Git. Cet outil nous a facilité le travail car il suffisait juste de faire un push sur Git et directement l’application étais mise à jour. La simplicité d’utilisation, la gratuité, la performance ainsi que la malléabilité ont été des ingrédients essentiels du choix de l’outil. En effet un certain temps de réflexion s’est imposé à nous entre : https://www.hubside.com/fr/, https://www.hostinger.fr/, https://www.heroku.com/. \\
\textbf{Node Geocoder } :
En commençant le projet, la première idée était d’utiliser l’api Google Maps, mais celle-ci étant devenue payante depuis juin 2018 et ne rentrant pas dans nos frais,  nous a obligé à trouver une tierce solution.  \\ La géolocalisation a donc été gérée grâce au module node Geocoder qui est une API OpenCage. Elle nous permet de soustraire un certains nombres de données de géolocalisation via les coordonnées gps ou via une adresse postale. 
\section{Implémentation}
\subsection{Explication du code}
\begin{tabular}{ | l | l | }
 \hline
 ./views & Contient les fichiers ejs qui seront transformés en Html \\ 
  \hline
  ./views/parials & Les elements communs entre toutes les pages (Menu,Header,Footer) \\ 
  \hline
 ./public/javascripts & Gére toute la partie javascript côté client  \\  
  \hline
 ./public/stylesheets & Contient les fichiers Css de l'application \\
  \hline
 ./routes & Contient toute la partie Javascript côté serveur.\\
  \hline
\end{tabular}
\\ \\
 Le projet a été implémenté grâce aux différents modules que pouvait nous fournir NodeJs. \\
 \textbf{Le contrôleur} :
 C’est le fichier ./app qui fait la liaison entre les différentes type de pages. \\
\textbf{Les routes} : Côté serveur  pages .js \\
\includegraphics{1} \\
\textbf{Les vues} :Côté client  pages.ejs \\
\includegraphics{2} \\
C’est également à ce niveau que nous défissions le dossier ./Public comme espace où nous allons stocker tous les documents annexes (feuille de style , javascript, images). \\
\includegraphics{3} \\
\textbf{La base de données} : Toujours dans le fichier ./app.js  nous nous connectons à la base de données qui est stockée en ligne, sur le site d’heroku. \\
\includegraphics{4} \\
\subsubsection{Inscription}
- L'utilisateur rempli le formulaire d'inscription et clique sur s'inscrire. \\
- Une requête Post est envoyée au serveur.\\
- Le serveur vérifie les informations saisies. \\
- Si les informations sont cohérentes le serveur hache le mot de passe et stocke les informations de l'utilisateur dans la BDD et renvoie un message de succès à l'utilisateur. \\
\subsubsection{Connexion}
- L'utilisateur tape ses informations de connexion et clique sur se connecter. \\
- Une requête post est envoyée au serveur qui vérifie l'existence de l'utilisateur en comparant l'email et le haché su mot de passe avec ceux qui sont dans la base de données.
- Si le compte existe le serveur crée une session et associe à cette session les informations du profil de l'utilisateur.

\subsubsection{Proposer un trajet}
- L'utilisateur rempli le formulaire avec les informations telle que l'adresse de départ et de destination ainsi que la date et heure de départ et le prix. \\
- Une requête Post est envoyée au serveur qui vérifie les données et insère le trajet dans la base de données. \\
\subsubsection{Rechercher un trajet}
- L’utilisateur rentre les informations dans le formulaire se situant dans la page ./views/rechercher.ejs \\
- Une requête Post est envoyée au serveur avec les informations saisies. \\
- Le serveur décode les adresses grâce au module Node Geocoder et cherche les trajets se situant à proximité de l'adresse de départ qui vont vers une adresse proche de l'adresse d'arrivée. \\
- Le serveur calcule le temps de trajet. \\
- Si l'heure d'arrivée à destination est inférieure à l'heure à laquelle l'utilisateur souhaite arriver le trajet est soumis à l'utilisateur.
- L'utilisateur a le choix ensuite de réserver sur un des trajets qui sont affichés lors de la recherche.
\subsubsection{Gestion des demandes} 
Correspond au moment où un utilisateur souhaite participer à un trajet. Ce dernier formule donc un souhait de participation vers le conducteur et c’est à celui que revient la charge d’accepter ou de refuser cette proposition. 
Pour faire le lien avec la section précédente (recherche d’un trajet), lorsqu’un participant  réserve un voyage, cette demande de réservation s’affiche dans la partie « Mes réservations » du conducteur et l’état affiché est « en attente ». 
Lorsque le conducteur veut répondre à la demande il clique sur le bouton en attente. Cette action entraine l’ouverture d’une modale qui prend en paramètre certains éléments essentiels concernant le trajet et les participants.	 \\
En fonction de la réponse du chauffeur deux actions différentes seront exécuté au niveau du serveur dans le fichier ./Routes/Repondre.js  \\
Si la réponse est positive  Accepter, on met à jour le nombre de places disponibles pour le trajet. 
\subsubsection{Messagerie} 
- Lorsque l'utilisateur consulte la page messagerie la liste des utilisateurs avec qui il a déjà discuté apparait.
- Le serveur créé une WebSocket et associe l'id du socket avec le compte de l'utilisateur.
- Lorsque l'utilisateur clique sur un autre utilisateur pour lancer la discussion avec lui, le client émet un signal "Afficher conversation" qui sera traité par le serveur. Ce dernier se charge récupérer les messages échangés entre les deux interlocuteurs et envoie le résultat à la page client pour que celle-ci les affiche.
- Lorsque l'utilisateur envoie un message,  le client émet un signal "new message" qui est traité au niveau du serveur. Ce dernier retourne donc un signal au socket associé à l'utilisateur destinataire du message.


\subsection{Securité de l'application}
De nos jours, maintenir la sécurité de nos applications est devenu primordiale afin d’éviter au maximum des attaques menaçantes. Mais avant de penser à la sécurité il faut dans premier temps connaitre ce qui nous menace. En d'autres termes quels sont les « failles » que peuvent contenir une application web. 
Dans ce qui suit, nous allons lister les failles les plus connues ainsi que les contres mesures prisent en prévention de possibles attaques.

\subsubsection{Les failles les plus connues}
\textbf{⦁ Injection SQL :} Ceci consiste à détourner le comportement de l’application en injectant du texte via les formulaires et donc modifier la requête exécutée. Par exemple on peut envoyer une requête qui nous enverra le mot de passe d’un utilisateur X ou alors afficher la liste de tous les clients ainsi que toutes leurs coordonnées. \\
\textbf{⦁ Attaque par force brute :} Sur les mots de passe en testant tout bêtement toutes les combinaisons possibles pour un mot de passe, jusqu’à l’obtention de la bonne combinaison. Ce type d’attaque dépend des performances de la machine ainsi que la taille du mot de passe. Cette attaque est réalisable en pratique tant que la complexité de calcul ne dépasse pas les 280 (d’après la loi de Moore). \\
\textbf{⦁ Les Failles XSS :} Ce sont des attaques faites par le biais des formulaires, en envoyant un bout de script HTML/JavaScript ayant une fonctionnalité malveillante ceci peut être fait en insérant directement le code dans un formulaire ou en le sauvegardant dans un fichier et faire appel à ce dernier. \\
\textbf{⦁ Les Failles include (ou URL) :} Beaucoup de sites en PHP utilisent un paramètre passé dans l’URL (en GET) pour afficher la page souhaitée et le hacker peut tout simplement modifier ce paramètre et il pourra rediriger le client vers un autre site malveillant. \\
\subsubsection{Les mesures de prévention}
- Élargir le domaine de caractère utilisé dans les mots de passe ainsi que leur longueur (utiliser des lettres minuscule, des lettres majuscule, des chiffres et des caractères spéciaux) permet d’augmenter la complexité d’attaque par force brute. \\
- À l’inscription, On stock la signature d’un mot de passe (nous avons utilisé la fonction de hachage sha256). Ce qui permet de ne pas connaitre le mot de passe lui-même et de l’utiliser pour l’authentification. Par exemple on compare la signature des mots de passe pour un utilisateur donné. \\
- Nous avons utilisé des requêtes préparées pour éviter les injections SQL et donc on récupère ce que l’utilisateur entre dans le formulaire, on le stock dans des variables qu’on insèrera dans la requête préparée. \\
- En utilisant NodeJs on évite l’utilisation du PHP coté serveur et ça nous évite les failles URL. \\
- Pour éviter les failles XSS on passe par une étape de vérification de paramètre en le récupérant du formulaire avant de l’envoyer. La vérification peut être sous forme de filtre (qui échappe les caractères spéciaux <, &, ”...etc.). \\
- A noter que, la sécurité d’un site web n’est pas garantie à 100\% car il peut toujours y avoir d’autre failles sur l’application qui peuvent être exploité par un hacker mais notre objectif est de réduire les faiblesses qu’ils peuvent y avoir.

\section{Produit livré}
\subsection{Accès Local
}
Le site est accessible de différentes manière. Dans un premier temps si vous le souhaitez, vous pouvez cloner le projet sur GitHub https://github.com/MaherAtt/CoivoitAws.
Ensuite lancez le projet en suivant ces étapes : \\
Npm install (permet de mettre à jour les différents modules utilisés) \\
Node bin/www pour lancer le projet
Puis ouvrez un navigateur web (Google Chrome de préférence) :  http://localhost:8000/
Si vous utilisez déjà le port 8000 pour un autre logiciel, vous pourrez bien évidemment le changer dans le fichier ./bin/www \\
\includegraphics{5} \\
À vous de jouer, suivez les étapes simples de d’accès (Inscription – Connexion – Ajouter trajet – Rechercher trajet (etc.) – Déconnexion) \\
À savoir que différents compte existent déjà si vous souhaitez uniquement voir les jeux de données : \\

\begin{tabular}{ | l | l | }
 \hline
 Email & Password \\ 
  \hline
  maher@gmail.com & maher \\ 
  \hline
  smail@gmail.com & smail  \\  
  \hline
 alao.a.s@gmail.com & alao \\
  \hline
\end{tabular} \\ \\
Si vous souhaitez voir la version mobile, il est préférable de suivre les instructions de la section suivante.

\subsection{Accès Online}
Le site est également accessible en ligne, cela permet donc de le voir sous différentes formes (taille d’écran).  
Connectez-vous sur ce site https://young-peak-38493.herokuapp.com/
La BDD étant hébergée en ligne, les identifiants de connexion restent les mêmes que ceux évoqués dans la section précédente.

\subsection{Récapitulatif et perspectives d'amélioration}
\begin{tabular}{ | l | l | }
\hline
Fonctionalité & Réalisée ? \\
\hline
Créer un compte &  Oui  \\ \hline
Se connecter & Oui   \\ \hline
Modifier les informations de l'utilisateur &   Oui  \\ \hline
Proposer un nouveau trajet &  Oui  \\ \hline
Rechercher un trajet &   Oui \\ \hline
Postuler et reserver sur un trajet & Oui   \\ \hline
Accepter/Refuser les demandes de participation aux trajets &  Oui  \\ \hline
Accéder aux profils des autres utilisateurs &  Oui  \\ \hline
Poster des avis et noter les utilisateurs &   Oui  \\ \hline
Rappel des trajets prévu pour chaque utilisateur &  Oui  \\ \hline
Suivre l'état des demandes de réservations &  Oui  \\ \hline
Communiquer avec les autres utilisateurs de l'application a travers une messagerie &  Oui  \\ \hline
Vérifier que l'email existe & Non \\ \hline
Vérifier que c'est un email universitaire & Non \\ \hline 

\end{tabular}

\section{Conclusion}
La réalisation de ce site web nous a tous appris. Il nous a fait gagner en compétences techniques (javascript, Node, Api maps etc.) Il nous également permit de tester nos capacités de réadaptation en fonction des différents aléas. En effet le confinement dû au corona virus nous a obligé à réfléchir et à se réorganiser autrement. 
La communication à distance est devenue primordiale car nous nous sommes retrouvés dans l’incapacité de nous réunir pour faire des points sur l’évolution du projet.
De plus il fallait trouver un moyen de s’entraider à distance et d’expliquer son point de vu, ce qui n’a pas toujours été chose simple. \\
Concernant le produit livré, nous sommes assez fières d’avoir pu vous livrer un site fonctionnel. \\
Certes se dernier comporte quelques  fonctionnalités auraient pû être optimisés (par exemple la notation des individus lorsqu'on souhaite aisser un avis). \\
Nous sommes aussi conscients qu'il y a denombreuses possibilitées d’améliorations à apporter au site web telle que la création de son application pour téléphone (sous android). 
Mais, réalisé dans des conditions quelques peut contraignantes, nous pensons et nous avons fait notre maximum afin d’atteindre au mieux les objectifs fixés. \\




\end{document}
